\chapter{Darknet Lite}
\label{chapter:Darknet}

As mentioned in Section \ref{sector:DeepVersat}, the DeepVersat system includes a RISC-V CPU to take out generic
code and to write the configuration runs into Versat's memories. This means the first step into implementing software
that can run any convolutional neural network on this system, the software must first run on the CPU then we offload Fixed Functions 
to Versat such as the convolutional layers, max pool, etc.

\section{Porting Darknet to an embedded CPU}

As mentioned in Section \ref{section:Darknet} is a framework for Neural Networks on C++ that uses dynamic memory 
and GPU acceleration option to get faster outputs.
Also, the use of floats is prohibited in the embedded code
as the RISC-V CPU only supports the extensions IM. I for Integer and M for multiplication.
It also has a lot of features that are not needed in this work, such as training the CNN.
By stripping the features of Darknet we get a much simpler
code framework appropriately named Darknet lite.

In the following figure, 
the data structure for a layer is shown. A CNN on Darknet lite is just an array of layers in which each has input, 
output, and layer parameters. 
Usually, the input is a past layer output or an image input.

\lstinputlisting[label=DarknetLiteStrut,language=C++,frame=single,breaklines=true,firstline=1,lastline=38,caption=Layer Struct
  Yolov3~\cite{yolov3}]{./Code/DarknetLiteStrut.h}

By Parsing the .cfg file, a configuration file is written in C with the layer array and static position of the data for each layer. 
Each Layer has its definition in C to be run by the embedded CPU but for the sake of this project, several layers can be
replaced by Functions that utilize Versat, the same way that the original Darknet framework had its functions written for CPU or GPU usage.

The following figure is an example of a CPU layer that computes the convolutional layer while using Fixed Point Logic.

\lstinputlisting[label=ConvolutionalLayerCPU,language=C++,frame=single,breaklines=true,firstline=54,lastline=85,caption=Convolutional Layer using only CPU and fixed memory]{./Code/convolutional_layer.c}



\section{Parsing CFG Files into the program}

Caffe~\cite{caffe} is a deep learning framework as shown in chapter \ref{chapter:Background},
using an open source tool~\cite{caffe2darknet}, the output can be set to CFG.
By using the network parser of Darknet, an array of layers is created with all
its required parameters. 

\lstinputlisting[label=listing:cfg2versat,language=C,frame=single,breaklines=true,firstline=22,lastline=61,caption=For Loop for writing darknet layers ]{./Code/parse2compiler.c}

Afterward, by going through each layer, "yolo.c" will be written
with all the data darknet lite will need. 
In listing \ref{listing:cfg2versat2}, the addresses of the data needed for the layer. 
In \ref{listing:cfg2versat3}, the static parameters are defined as well.

\newpage
\lstinputlisting[label=listing:cfg2versat2,language=C,frame=single,breaklines=true,firstline=15,lastline=21,caption=For Loop for writing darknet layers ]{./Code/yolo.c}

\lstinputlisting[label=listing:cfg2versat3,language=C,frame=single,breaklines=true,firstline=133,lastline=139,caption=For Loop for writing darknet layers ]{./Code/yolo.c}
