\chapter{Darknet Lite}
\label{chapter:Darknet}


The DeepVersat system, as described in Section \ref{sector:DeepVersat}, incorporates a RISC-V CPU responsible for executing generic code and storing configuration runs in Versat's memories. 
This establishes the critical requirement of ensuring compatibility between the embedded CPU and the framework to enable the execution of diverse convolutional neural networks on the system. 
Furthermore, for optimized performance, the system delegates the execution of fixed functions, such as the convolutional layers, to Versat.

\section{Porting Darknet to an embedded CPU}

As mentioned in Section \ref{section:Darknet} is a framework for Neural Networks on C++ that uses dynamic memory and GPU acceleration option to get faster outputs.
Also, the use of floats is prohibited in the embedded code as the RISC-V CPU only supports the extensions IM. I for Integer and M for multiplication.
It also has a lot of features that are not needed in this work, such as training the CNN.
By stripping the features of Darknet we get a much simpler code framework appropriately named Darknet lite.

In the following figure, the data structure for a layer is shown. 
A CNN on Darknet lite is just an array of layers in which each has input, output, and layer parameters. 
Usually, the input is a past layer output or an image input.

\lstinputlisting[label=DarknetLiteStrut,language=C++,frame=single,breaklines=true,firstline=1,lastline=38,caption=Layer Struct
  Yolov3~\cite{yolov3}]{./Code/DarknetLiteStrut.h}

By Parsing the .cfg file, a configuration file is written in C with the layer array and static position of the data for each layer. 
Each Layer has its definition in C to be run by the embedded CPU, but for the sake of this project, several layers can be replaced by functions that utilize Versat, the same way that the original Darknet framework had its functions written for CPU or GPU usage.

The following Listingis an example of a CPU layer that computes the convolutional layer while using Fixed Point Logic.

\lstinputlisting[label=ConvolutionalLayerCPU,language=C++,frame=single,breaklines=true,firstline=54,lastline=85,caption=Convolutional Layer using only CPU and fixed memory]{./Code/convolutional_layer.c}



\section{Parsing CFG Files into the program}

Caffe~\cite{caffe} is a deep learning framework as shown in chapter \ref{chapter:Background}, using an open source tool~\cite{caffe2darknet}, the output can be set to CFG.
By using the network parser of Darknet, an array of layers is created with all its required parameters. 

\lstinputlisting[label=listing:cfg2versat,language=C,frame=single,breaklines=true,firstline=22,lastline=61,caption=For Loop for writing darknet layers ]{./Code/parse2compiler.c}

Afterward, by going through each layer, "yolo.c" will be written with all the data darknet lite will need. 
In Listing\ref{listing:cfg2versat2}, the addresses of the data needed for the layer. 
In \ref{listing:cfg2versat3}, the static parameters are defined as well.

\lstinputlisting[label=listing:cfg2versat2,language=C,frame=single,breaklines=true,firstline=15,lastline=21,caption=For Loop for writing darknet layers ]{./Code/yolo.c}

\lstinputlisting[label=listing:cfg2versat3,language=C,frame=single,breaklines=true,firstline=133,lastline=139,caption=For Loop for writing darknet layers ]{./Code/yolo.c}
