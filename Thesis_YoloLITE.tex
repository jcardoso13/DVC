\chapter{Darknet Lite}
\label{chapter:Darknet}

As mentioned in Section \ref{chapter:DeepVersat}, the Deep Versat system includes a RISC-V CPU to take out generic
code and to write the configuration runs into Versat's memories. This means the first step into implemetning software
that can run any convolutional neural network on this system, it must first run on the CPU then we off load Fixed Functions to Versat such as the convolutional layers, maxpool etc.

\section{Porting Darknet to an embedded CPU}

As mentioned in Section \ref{section:darknet} is a framework for Neural Networks on C++ that uses dynamic memory and GPU acceleration option to get faster outputs. Also the use of floats is also prohibited in the embedded code
as the RISC-V CPU only supports the extentions IM. I for Integer and M for multiplication. It also has a lot of features that are not needed in this work, such as training the CNN. By stripping the features of darknet we get a much simpler
code framework apropriately named darknet lite.

In the following figure, the data strucuture for a layer is showed. A CNN on darknet lite is just an array of layers in which each has an input, output and layer parameters. Usually the input is a past layer output or the image input.

\lstinputlisting[label=DarknetLiteStrut,language=C++,frame=single,breaklines=true,firstline=1,lastline=38,caption=Layer Struct
  Yolov3~\cite{yolov3}]{./Code/DarknetLiteStrut.h}


\section{Conversion of Caffe to CFG}



\section{Writting Layers into Memory}