%%%%%%%%%%%%%%%%%%%%%%%%%%%%%%%%%%%%%%%%%%%%%%%%%%%%%%%%%%%%%%%%%%%%%%%%
%                                                                      %
%     File: Thesis_Resumo.tex                                          %
%     Tex Master: Thesis.tex                                           %
%                                                                      %
%     Author: Andre C. Marta                                           %
%     Last modified :  2 Jul 2015                                      %
%                                                                      %
%%%%%%%%%%%%%%%%%%%%%%%%%%%%%%%%%%%%%%%%%%%%%%%%%%%%%%%%%%%%%%%%%%%%%%%%

\section*{Resumo}

% Add an entry in the table of contents as a section
\addcontentsline{toc}{section}{Resumo}

% Esta tese apresenta uma solução para simular o DeepVersat, uma CGRA que é acupulada de um processador RISC-V.
% Tambem é apresentada nesta tese ferramentas para o DeepVersat correr Redes Neuronais Convolucionais. Estas cargas de trabalho são
% usadas em diversos algoritmos de Inteligencia Arteficial como a deteção de objetos em imagens. As vantagens da ferramenta são várias.
% Primeiro, a escrita das configurações do Versat é preciso conhecimento da arquitetura a nivel detalhado e das suas APIs.
% Segundo, a escrita de algoritmos complexos para o Versat é preciso muitas horas de desenvolvimento e mais outras quantas para testar em hardware, ou seja
% o custo de usar o Versat baixa consideravelmente e a performance é otimizada á configuração do Versat escolhida podendo 
% testar centenas de configurações de Hardware para otimizar a performance de uma rede em especifico.

O foco desta tese concentra-se na aceleração de Redes Neuronais Profundas (DNN) com os recursos da matriz reconfigurável de grão grosso (CGRA) DeepVersat.
O objetivo principal é desenvolver um compilador que converta as descrições DNN em código executável otimizado para o sistema CPU/DeepVersat. Para conseguir isso, uma estrutura de rede neuronais, Darknet, é estendida, adaptada e simpleficada para compilar arquivos de descrição DNN em código que integra-se com o sistema, utilizando a interface de software (API) do Versat. A API do Versat foi expandida para conseguir a aceleração de camadas de computação intesiva, com alocação dinâmica de recursos para melhorar o desempenho. O simulador em software também foi desenvolvido para facilitar a otimização  arquitetônica e reduzir o tempo de desenvolvimento para implementações baseadas no DeepVersat. A utilidade do Darknet Lite na compilação de DNNs no código Versat e a eficácia da nova API em várias condigurações de hardware são demonstradas por vários ficheiros de teste, estabelecendo uma prova de conceito para a abordagem proposta.


\vfill

\textbf{\Large Palavras-chave:} Matrizes Reconfiguráveis de Grão Grosso, Versat, Darknet, Redes Neuronais Convolucionais, Redes Neuronais Profundas, Simulador, Sistemas Heterógenos, Sistemas Embebidos

