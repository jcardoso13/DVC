%%%%%%%%%%%%%%%%%%%%%%%%%%%%%%%%%%%%%%%%%%%%%%%%%%%%%%%%%%%%%%%%%%%%%%%%
%                                                                      %
%     File: Thesis_Resumo.tex                                          %
%     Tex Master: Thesis.tex                                           %
%                                                                      %
%     Author: Andre C. Marta                                           %
%     Last modified :  2 Jul 2015                                      %
%                                                                      %
%%%%%%%%%%%%%%%%%%%%%%%%%%%%%%%%%%%%%%%%%%%%%%%%%%%%%%%%%%%%%%%%%%%%%%%%

\section*{Resumo}

% Add an entry in the table of contents as a section
\addcontentsline{toc}{section}{Resumo}

% Esta tese apresenta uma solução para simular o DeepVersat, uma CGRA que é acupulada de um processador RISC-V.
% Tambem é apresentada nesta tese ferramentas para o DeepVersat correr Redes Neuronais Convolucionais. Estas cargas de trabalho são
% usadas em diversos algoritmos de Inteligencia Arteficial como a deteção de objetos em imagens. As vantagens da ferramenta são várias.
% Primeiro, a escrita das configurações do Versat é preciso conhecimento da arquitetura a nivel detalhado e das suas APIs.
% Segundo, a escrita de algoritmos complexos para o Versat é preciso muitas horas de desenvolvimento e mais outras quantas para testar em hardware, ou seja
% o custo de usar o Versat baixa consideravelmente e a performance é otimizada á configuração do Versat escolhida podendo 
% testar centenas de configurações de Hardware para otimizar a performance de uma rede em especifico.

Nesta tese, o problema de acelerar a execução de Redes Neuronais (DNNs) usando CGRA com emfase em compilar uma descrição de uma DNN para codigo que corre num sistema
CPU/CGRA. Este é um tópico vasto, por isso esta tese foca-se na simulação do DeepVersat, um sistema CPU/CGRA capaz de correr DNNs, e ferramentas para correr qualquer
tipo de Rede Neuronais Convolucionais (CNNs) no DeepVersat. As ferramentas apresentadas nesta tese dão a possibilidade de exploração e otimização, e uma redução dramática
de tempo de desenvolvimento.


\vfill

\textbf{\Large Palavras-chave:} CGRA, Versat, Darknet, Redes Neuronais

