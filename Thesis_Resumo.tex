%%%%%%%%%%%%%%%%%%%%%%%%%%%%%%%%%%%%%%%%%%%%%%%%%%%%%%%%%%%%%%%%%%%%%%%%
%                                                                      %
%     File: Thesis_Resumo.tex                                          %
%     Tex Master: Thesis.tex                                           %
%                                                                      %
%     Author: Andre C. Marta                                           %
%     Last modified :  2 Jul 2015                                      %
%                                                                      %
%%%%%%%%%%%%%%%%%%%%%%%%%%%%%%%%%%%%%%%%%%%%%%%%%%%%%%%%%%%%%%%%%%%%%%%%

\section*{Resumo}

% Add an entry in the table of contents as a section
\addcontentsline{toc}{section}{Resumo}

% Esta tese apresenta uma solução para simular o DeepVersat, uma CGRA que é acupulada de um processador RISC-V.
% Tambem é apresentada nesta tese ferramentas para o DeepVersat correr Redes Neuronais Convolucionais. Estas cargas de trabalho são
% usadas em diversos algoritmos de Inteligencia Arteficial como a deteção de objetos em imagens. As vantagens da ferramenta são várias.
% Primeiro, a escrita das configurações do Versat é preciso conhecimento da arquitetura a nivel detalhado e das suas APIs.
% Segundo, a escrita de algoritmos complexos para o Versat é preciso muitas horas de desenvolvimento e mais outras quantas para testar em hardware, ou seja
% o custo de usar o Versat baixa consideravelmente e a performance é otimizada á configuração do Versat escolhida podendo 
% testar centenas de configurações de Hardware para otimizar a performance de uma rede em especifico.

O objetivo deste trabalho é estender os recursos do DeepVersat, que é uma arquitetura em matrizes reconfiguráveis de grão grosso (CGRA), para
processar Redes Neurais Profundas (DNN), com ênfase particular na compilação de uma descrição DNN em código que é executado num
Sistema CPU/DeepVersat.
Primeiro, para passar de um ficheiro de descrição DNN para um código executável, uma "framework" de rede neuronais deve ser adaptada para executar as diferentes camadas
no Versat para aceleração. Após análise das possibilidades, o Darknet é visto como uma escolha clara, pois é uma "framework" de código aberto escrito em C, compatível
com a interface de programação do Versat. Primeiro, no entanto, a estrutura precisa de ser adaptada e reduzida para o uso pretendido, criando no processo o Darknet Lite.
A interface do Versat cresce para alcançar a aceleração das camadas de computação intensiva, adicionando mais camadas de abstração para alocar os recursos
da configuração de hardware implementada em tempo real para trazer o mais alto desempenho possível. Além disso, um simulador do DeepVersat permite otimização arquitetônica e reduz drasticamente o tempo de desenvolvimento
para correr o DeepVersat com base no arquivo de configuração de hardware.
Esta dissertação examina o problema de acelerar a execução de DNNs usando CGRAs juntamente com a compilação de DNNs em FPGAs. As iterações Versat CGRA e DeepVersat são explicadas em detalhes.
Além disso, apresenta o Darknet Lite, o simulador do DeepVersat e a nova interface. Por fim, é apresentada uma série de aplicações para testar o simulador e a nova interface que mostram o potencial
do simulador, fornecendo o número de iterações necessárias para executar uma camada convolucional numa determinada configuração do DeepVersat. O utilizador, então, pode ajustar a configuração para estudar o desempenho
e escolher a configuração mais eficaz e eficiente para esse DNN.


\vfill

\textbf{\Large Palavras-chave:} Matrizes Reconfiguráveis de Grão Grosso, Versat, Darknet, Redes Neuronais Convolucionais, Redes Neuronais Profundas, Simulador, Sistemas Heterógenos, Sistemas Embebidos

