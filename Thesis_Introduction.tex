%%%%%%%%%%%%%%%%%%%%%%%%%%%%%%%%%%%%%%%%%%%%%%%%%%%%%%%%%%%%%%%%%%%%%%%%
%                                                                      %
%     File: Thesis_Introduction.tex                                    %
%     Tex Master: Thesis.tex                                           %
%                                                                      %
%     Author: Andre C. Marta                                           %
%     Last modified :  2 Jul 2015                                      %
%                                                                      %
%%%%%%%%%%%%%%%%%%%%%%%%%%%%%%%%%%%%%%%%%%%%%%%%%%%%%%%%%%%%%%%%%%%%%%%%

\chapter{Introduction}
\label{chapter:introduction}


FIX: mete citaçoes nas coisas mais importantes


In this report, the problem of accelerating the execution of Deep Neural
Networks (DNNs) using Coarse GRained Reconfigurable Arrays (CGRAs) is studied,
with special emphasis on compiling a DNN description into code that runs on
CPU/CGRA system. The Deep Versat Architecture~\cite{bla} CGRA will be used as an
implementation tool in this work.


%%%%%%%%%%%%%%%%%%%%%%%%%%%%%%%%%%%%%%%%%%%%%%%%%%%%%%%%%%%%%%%%%%%%%%%%
\section{Problem}
\label{section:problem}

Neural Networks have been an object of study since the 1940's, but until the
beginning of this decade their applications were limited and did not play a
major role in computer vision conferences. With its meteoric rise in research,
several solutions to accelerate this algorithm have appeared, from FIX (FPGA) to
FIX (ASIC) implementations.

Convolutional Neural Networks (CNNs) are a particular kind of DNN where the output
values of the neurons in one layer are convolved with a kernel to produce the
input values of the neurons of the next layer. This algorithm is compute bound,
that is, its performance depends on how fast it can do certain calculations, and
depend less on the memory access time. Namely the convolutional layers take
approximately 90$\%$ of the computation time.

The acceleration of these workloads is a matter of importance for today's
applications such as image processing for object recognition or simply to
enhance certain images. Other uses like instant translation and virtual
assistants are applications of neural networks and their acceleration is of
vital importance to bring them into Internet of Things.

A suitable circuit to accelerate DNNs in hardware is the CGRA. A CGRA is a
collection of Functional Units and memories with programmable interconnections
in order to form computational datapaths. A CGRA can be implemented in both
FPGAs and ASICs. CGRAs can be reconfigured much faster than FPGAs, as they have
much less configuration bits. If reconfiguration is done at runtime, CGRAs add
temporal scalability to the spacial scalabilty that characterize
FPGAs. Moreover, partial reconfiguration is much easier to do in CGRAs compared
to FPGAs which further speeds up reconfiguration time. Another advantage of
CGRAs is the fact that they can be programmed entirely in software, contrasting
with the large development time of customized Intellectual Property (IP) blocks.
The Coarse Grain Reconfigurable Arrays (CGRA) is a midway acceleration solution
between FPGAs, which are flexible but large, power hungry and difficult to
reprogram, and ASICs, which are fast but generally not programmable.

However, mapping a specific DNN to a CGRA requires knowledge of its
architecture, latencies and register configurations, which may become a lengthy
process, especially if the user wants to explore the design space for several
DNN configurations. An automatic compiler that can map a standard DNN
description into CPU/CGRA code would dramatically decrease time to market of its
users. Currently there are equivalent tools for CPUs and GPUs and
even for FPGAS.


%%%%%%%%%%%%%%%%%%%%%%%%%%%%%%%%%%%%%%%%%%%%%%%%%%%%%%%%%%%%%%%%%%%%%%%%
\section{Solution}
\label{section:solution}

The proposed solution is a compiler that takes a configuration file from a
neural network framework like Caffe or Darknet. This new tool inputs the
parameters of Deep Versat, such as the number of layers and functional units,
and produces the C code needed for the Versat runs. This code is run on the
RISC-V picorv32~\cite{bla} CPU controller that has Deep Versat as a peripheral.


%%%%%%%%%%%%%%%%%%%%%%%%%%%%%%%%%%%%%%%%%%%%%%%%%%%%%%%%%%%%%%%%%%%%%%%%
%\section{Thesis Outline}
%\label{section:outline}

%Briefly explain the contents of the different chapters...

%%%%%%%%%%%%%%%%%%%%%%%%%%%%%%%%%%%%%%%%%%%%%%%%%
%\section{Author's Work}
%\label{section:authorwork}

%TO ADD----

%%%%%%%%%%%%%%%%%%%%%%%%%%%%%%%%%%%%%%%%%%%%%%%%%%
\section{Report Outline}
\label{reportoutline}

This report is composed of 4 more chapters. In the second chapter, the
state-of-the-art of neural networks and the difficulties accelerating them is
described. In the third chapter, the Deep Versat architecture and how to program
it is explained. In the fourth chapter, CNN compiler techniques are
explored. Finally, the last chapter contains the proposed solution and the plan
for its execution.


