%%%%%%%%%%%%%%%%%%%%%%%%%%%%%%%%%%%%%%%%%%%%%%%%%%%%%%%%%%%%%%%%%%%%%%%%
%                                                                      %
%     File: Thesis_Abstract.tex                                        %
%     Tex Master: Thesis.tex                                           %
%                                                                      %
%     Author: Andre C. Marta                                           %
%     Last modified:  2 Jul 2015                                      %
%                                                                      %
%%%%%%%%%%%%%%%%%%%%%%%%%%%%%%%%%%%%%%%%%%%%%%%%%%%%%%%%%%%%%%%%%%%%%%%%

\section*{Abstract}

% Add entry in the table of contents as section
\addcontentsline{toc}{section}{Abstract}

This thesis presents a solution to simulate Deep Versat, a CGRA, which is coupled to a RISC-V CPU.
It is also presented in this thesis the tools for Deep Versat to run any Convolutional Neural Network with any configuration of datapaths.
These workloads are used in Machine Learning algorithms with object detention in images. The tool has several advantages.
Firstly, in the configuration writing to the Registers, there's a need for a high degree of knowledge of the architecture of the CGRA and its software
APIs.
Secondly, the writing of complex algorithms on this hardware needs long hours of debugging and development, meaning by the use of the tools
presented in this thesis, the development time can be reduced and performance can be automatically optimized.
Finally, the tools can be adapted to changes in the hardware by changing a few software functions at most.

\vfill

\textbf{\Large Keywords:} CGRA, Versat, Darknet, Convolutional Neural Networks, Deep Neural Networks


