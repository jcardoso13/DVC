%%%%%%%%%%%%%%%%%%%%%%%%%%%%%%%%%%%%%%%%%%%%%%%%%%%%%%%%%%%%%%%%%%%%%%%%
%                                                                      %
%     File: Thesis_Abstract.tex                                        %
%     Tex Master: Thesis.tex                                           %
%                                                                      %
%     Author: Andre C. Marta                                           %
%     Last modified:  2 Jul 2015                                      %
%                                                                      %
%%%%%%%%%%%%%%%%%%%%%%%%%%%%%%%%%%%%%%%%%%%%%%%%%%%%%%%%%%%%%%%%%%%%%%%%

\section*{Abstract}

% Add an entry in the table of contents as a section
\addcontentsline{toc}{section}{Abstract}


This thesis focuses on accelerating Deep Neural Networks (DNN) with the capabilities of the DeepVersat Coarse-Grained Reconfigurable Array (CGRA). The primary objective is to develop a compilation approach that converts DNN descriptions into executable code optimized for CPU/DeepVersat system. To achieve this, a neural network framework, Darknet, is extended, adapted, and streamlined to compile DNN description files into code that integrates with the system, utilizing the Versat Application Programming Interface (API). The Versat API is expanded to enable acceleration of compute-intensive layers, with dynamic resource allocation for improved performance. A software simulator is also developed to facilitate architectural optimization and reduce development time for DeepVersat-based implementations. 
The usefulness of Darknet Lite in compiling DNNs into Versat code and the effectiveness of the new API on various hardware configurations are demonstrated through multiple test files, establishing a proof of concept for the proposed approach.

\vfill

\textbf{\Large Keywords:} Coarse-Grained Reconfigurable Array, Versat, Convolutional Neural Networks, Deep Neural Networks, Simulator, Heterogeneous Systems
